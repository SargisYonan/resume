%% Resume
% @author Sargis Yonan

%% PREAMBLE
\documentclass[letterpaper,11pt,oneside]{article}

\usepackage{fontspec}
\usepackage[english]{babel}
\setmainfont[Mapping=text-tex]{Arial}

%% Sectioning commands
% everything for inverted section title boxes
\usepackage{xcolor,keyval}
\makeatletter
\definecolor{blackbg}{rgb}{0.0,0.0,0.0}
\newlength{\@mcb@width}
\define@key{mcb}{bg}{\def\@mcb@bg{#1}}
\define@key{mcb}{width}{\setlength\@mcb@width{#1}}
\newcommand{\sectioninverttitle}[2][]{{%
    \setkeys{mcb}{width=0.5\linewidth,bg=blackbg,#1}%
    \colorbox{\@mcb@bg}{\begin{minipage}{\@mcb@width}
    \strut#2\strut
  \end{minipage}}
}}
\makeatother

%% Section title code (black background, white text)
\newcommand{\sectiontitle}[1]{\sectioninverttitle[width=0.135\textwidth]{\textcolor{white}{\textbf{\normalsize{#1}}}}}

\usepackage{setspace}
\usepackage{hyperref}

%% Bullet pointing commands
\usepackage{amssymb}
\newcommand{\blacksquarebullet}{\tiny{\ensuremath{\blacksquare}}}
\newcommand{\squareitem}[1]{\item[\blacksquarebullet]{#1}}
\usepackage{enumitem}
\setlist{nosep}

%% Margins settings
\usepackage[left=0.4in, right=0.5in, bottom=0.5in, top=0.5in, headheight=14pt, footskip=14pt]{geometry}
\usepackage[geometry]{ifsym}

\def \tablefillwidth {0.98\linewidth}

%% For overall table format
\usepackage{longtable}

%Changes the page numbers - {arabic}=arabic numerals, {gobble}=no page numbers, {roman}=Roman numerals
\pagenumbering{gobble}

\usepackage{fancyhdr}
\pagestyle{fancy}
\renewcommand{\headrulewidth}{0pt}
\lfoot{\footnotesize{\textcolor{gray}{Sargis Yonan}}}
%% END OF PREAMBLE 

\begin{document}

%% HEADER NAME
\noindent {\Huge{\textbf{Sargis Yonan}}} \\
\normalsize \vspace{1em}
\noindent Computer, Software, and Robotics Engineer

\vspace{0.2in}

%%% BEGIN LONG TABLE
\begin{longtable}{@{\extracolsep{\fill}} p{0.15\textwidth} p{0.8\textwidth}}

%% CONTACT
\sectiontitle{CONTACT}
& \hspace*{\fill}
\begin{tabular}[t]{l r}
    \textbf{Email}: & \href{mailto:sargis@yonan.org}{sargis@yonan.org} \\

    \textbf{Website}: & \href{https://www.yonan.org}{sargis.yonan.org}\\

    \textbf{LinkedIn}: & \href{https://www.linkedin.com/in/sargisyonan}{linkedin.com/in/sargisyonan}\\

    \textbf{GitHub}: & \href{https://github.com/SargisYonan}{github.com/SargisYonan}\\

\end{tabular}
\vspace{1em}
\\

%% EXPERIENCE
\sectiontitle{WORK\\EXPERIENCE}
&
%% Apple
\begin{tabular}[t]{@{\extracolsep{\fill}} p{\tablefillwidth}}
{\large{\textbf{Apple Inc.}}} \hspace*{\fill} \textbf{March 2019 - Present}
\\
Camera Systems Engineer \\\\
\begin{enumerate}[leftmargin=*]
    \squareitem Develop and maintain XNU kernel drivers and userspace software, including frameworks and testing tools for Apple’s camera technologies.
    \squareitem Design camera frameworks and client-facing APIs so apps and system services can reliably access imaging features across Apple platforms.
    \squareitem Collaborate with image signal processing teams to architect interprocess communications and commanding interfaces, optimize system performance, and ensure feature compatibility.
    \squareitem Lead hardware bring-up efforts, working extensively with ARM SoCs on device trees, bootloaders, the OS kernel, and userspace on Apple’s platforms.
    \squareitem Have contributed and shipped dozens of major features and low-level camera system software stacks across all Apple devices with cameras.
    \squareitem Work closely with the factory team to ensure successful production ramping across hardware build phases and lifecycles.
    \squareitem Built and optimized system-level high-performance, low-latency data delivery across the camera software and hardware stack.
\end{enumerate}
\end{tabular} \\

%% SpaceX
\\ &
\begin{tabular}[t]{@{\extracolsep{\fill}} p{\tablefillwidth}}
{\large{\textbf{SpaceX}}} \hspace*{\fill} \textbf{June 2018 - September 2018}
\\
Associate Flight Software Engineer, Starlink \\\\
\begin{enumerate}[leftmargin=*]
    \squareitem Developed firmware for early-stage Starlink satellites, working on ARM-based systems with an RTOS.
    \squareitem Designed fault-tolerant, system-critical code for sensor and memory drivers, contributing to reliable operation in flight.
    \squareitem Implemented a time synchronization algorithm for the flight computer system, enhancing communication accuracy across satellite subsystems.
\end{enumerate}
\end{tabular} \\

%% UCSC
\\ &
\begin{tabular}[t]{@{\extracolsep{\fill}} p{\tablefillwidth}}
{\large{\textbf{University of California, Santa Cruz}}} \hspace*{\fill} \textbf{January 2018 - December 2018}
\\
Teaching Assistant\\\\
\textbf{Microprocessor Systems Design} (Fall 2018):  
Led labs on embedded systems, C programming, hardware debugging, and Linux-based software design.\\

\textbf{Engineering Capstone Design} (Winter - Spring 2018):  
Advised 20 senior project teams, assisting in system design, architecture, debugging, and project execution across diverse domains, including robotics and mesh networking.\\
\end{tabular} \\

\\ &
\begin{tabular}[t]{@{\extracolsep{\fill}} p{\tablefillwidth}}
{\large{\textbf{CityBlooms Urban Micro Farms}}} \hspace*{\fill} \textbf{June - October 2017, January - March 2019}
\\
Embedded Software and Hardware Engineering Intern \\\\

Developed sensor and peripheral drivers in C/C++ for microcontrollers and Linux IoT systems, automating deployments for live agricultural environments.
\end{tabular} \\

\\ &
\begin{tabular}[t]{@{\extracolsep{\fill}} p{\tablefillwidth}}
{\large{\textbf{Pearl Automation}}} \hspace*{\fill} \textbf{January - December 2016}
\\
Firmware Engineering Intern \\\\

Created software for ARM controllers and Embedded Linux. Developed a dynamic frequency scaling algorithm that significantly improved device battery life.
\end{tabular} \\

\vspace{1em}
\\

%% EDUCATION
\sectiontitle{EDUCATION}
&
%% Masters
\begin{tabular}[t]{@{\extracolsep{\fill}} p{\tablefillwidth}}
    {\large{\textbf{M.Sc. in Robotics Engineering}}} \hspace*{\fill} \textbf{September 2017 - December 2018} \\

    Emphasis: Robotics Engineering \& Control Theory \\
    University of California, Santa Cruz \\
    \begin{enumerate}[leftmargin=*]
        \squareitem Courses: Robotics, Optimal Estimation, Stochastic Filtering, Control Theory
        \squareitem Thesis: \href{https://www.yonan.org/masters_thesis}{Improved Field Exploration with Variance Suppressing Path Planning}
        \squareitem Developed path planning techniques for optimal autonomous field exploration
    \end{enumerate}
\end{tabular} \\

&
%% Bachelors
\begin{tabular}[t]{@{\extracolsep{\fill}} p{\tablefillwidth}}
    {\large{\textbf{B.S. in Computer Engineering}}} \hspace*{\fill} \textbf{September 2013 - June 2017} \\

    Concentration: Robotics \& Control \\
    University of California, Santa Cruz \\
    \begin{enumerate}[leftmargin=*]
        \squareitem Research/Projects: UAVs, Sensor Networking, Power Systems
        \squareitem Graduated with Honors in Major
    \end{enumerate}
\end{tabular} \\

&
%% Minor
\begin{tabular}[t]{@{\extracolsep{\fill}} p{\tablefillwidth}}
    {\large{\textbf{Minor in Computer Science}}} \hspace*{\fill} \textbf{September 2013 - June 2017} \\

    University of California, Santa Cruz \\
\end{tabular}

\vspace{1em}
\\

%% PROJECTS
\sectiontitle{PROJECTS}
&
\begin{tabular}[t]{@{\extracolsep{\fill}} p{\tablefillwidth}}

%% uLAPack
\href{https://github.com/SargisYonan/uLAPack}{\textbf{Micro Linear Algebra Package}} - A matrix math library optimized for embedded systems. The library contains functionality for learning, data fitting, filtering, and controls.\\
\href{https://github.com/SargisYonan/uLAPack}{\scriptsize github.com/SargisYonan/uLAPack}\\

%% uKal
\href{https://github.com/SargisYonan/ukal}{\textbf{Embedded Kalman Filter}} - A Kalman filtering library for linear and nonlinear systems. The library is ideal for embedded systems where dynamic memory allocation is a concern.\\
\href{https://github.com/SargisYonan/ukal}{\scriptsize github.com/SargisYonan/ukal}

%% Masters
\href{https://github.com/SargisYonan/field_exploration}{\textbf{Field Exploration Simulation Framework}} - A MATLAB framework and simulation environment for autonomous field exploration using prediction variance suppression techniques introduced in my Masters thesis.\\
\href{https://github.com/SargisYonan/field_exploration}{\scriptsize github.com/SargisYonan/field\_exploration}\\

\end{tabular}

\vspace{1em}
\\

%% Skills
\sectiontitle{KEY SKILLS}
&
\begin{tabular}[t]{@{\extracolsep{\fill}} p{\tablefillwidth}}
\textbf{Programming}: C, C++, Python, XNU Kernel Development \\
\textbf{Embedded Systems}: ARM SoCs, AVR, Firmware Development \\
\textbf{Robotics}: Sensor Fusion, Feedback Control, Sensor Design \\
\textbf{Math}: Kalman Filters, Control Theory, Geostatistics \\
\end{tabular}
\vspace{1em}
\\

\end{longtable}

\end{document}
